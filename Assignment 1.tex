\documentclass[12pt,letterpaper]{article}
\usepackage[utf8]{inputenc}
\usepackage[english]{babel}
\usepackage{amsmath}
\usepackage{amsfonts}
\usepackage{amssymb}
\usepackage{graphicx}
\usepackage[left=2cm,right=2cm,top=2cm,bottom=2cm]{geometry}
\author{Andrew Fillmore}
\title{Assignment 1}
\begin{document}
\maketitle
\section*{Problem 6}
\subsection*{a}

$d_{prop}=m/s$
\subsection*{b}

$d_{trans}=L/R$
\subsection*{c}

The end-to-end delay will be $d_{total}$.
Then $d_{total}=d_{prop}+d_{trans}$
\subsection*{d}
At time $t=d_{trans}$ the last bit of the packet will be being transmitted from Host A.
\subsection*{e}

If $d_{prop}>d_{trans}$ then at time $t=d_{trans}$ the first bit will be at $d_{trans}\times s$ meters.
\subsection*{f}

If $d_{prop}<d_{trans}$ then at time $t=d_{trans}$ the first bit will have arrived at Host B already.
\subsection*{g}

If
$$d_{trans}=L/R=120/56000=3/1400$$
then
$$d_{prop}=3/1400.$$
Because
$$d_{prop}=m/s\Rightarrow m=d_{prop}\times s$$
then
$$m=(3/1400)\times2.5\times10^{8}=535,714.28571428571428571428571429meters$$
\section*{10}

The total end-to-end delay for the packets will be
$$\frac{d_1}{s_1}+\frac{L}{R_1}+d_{proc}+\frac{d_2}{s_2}+\frac{L}{R_2}+d_{proc}+\frac{d_3}{s_3}+\frac{L}{R_3}.$$
With values the delay will be
$$\frac{5000km}{2.5\times10^8m/s}+\frac{1500b}{2Mbps}+3ms+\frac{4000km}{2.5\times10^8m/s}+\frac{1500b}{2Mbps}+3ms+\frac{1000km}{2.5\times10^8m/s}+\frac{1500b}{2Mbps}$$
$$=0.02s+0.75ms+3ms+0.016s+0.75ms+3ms+0.004s+0.75ms$$
$$=48.3ms$$
\section*{11}

Assumption: This problem is asking for the general form, not using the specific values from the previous problem.

The end-to-end delay for this new system will be
$$\frac{L}{R}+\frac{d_1}{s_1}+\frac{d_2}{s_2}+\frac{d_3}{s_3}.$$
\section*{18}

\section*{23}
\subsection*{a}

The inter-arrival time will be $\frac{L}{R_s}+\frac{L}{R_c}$ seconds.
This is assuming the router is using store-forwarding.
If the router is not using store-forwarding then the inter-arrival time will be $2\frac{L}{R_s}$ seconds.
\subsection*{b}

If the router is using store-forwarding then there will always be a delay while waiting for the entire packet to arrive. Even if the router is not using store-forwarding the second packet will be queued for $\frac{L}{R_c}-\frac{L}{R_s}$ seconds.
This is the difference between the time each link takes to transmit a packet.
Therefore $T$ must be $\frac{L}{R_c}-\frac{L}{R_s}$ to ensure no queuing.
\section*{24}
40 terabytes is $3.2\times10^{14}$ bits.
It will take $3.2\times10^6$ seconds to transmit which is equivalent to 37 days 5 minutes 20 seconds.
Therefore it would be better to send the data by FedEx.

``Never underestimate the bandwidth of a station wagon full of tapes hurtling down the highway." - Andrew S. Tanenbaum

\section*{31}
\subsection*{a}

It will take 4 seconds to move the message from the source host to the first packet switch.
Assuming that there are 2 switches between the source and the the host, it will take 12 seconds to move the message from source host to destination host.
\subsection*{b}

It will take 5ms to move the first packet from source host to the first switch.
The second packet will be fully received by the first switch at 10ms.
\subsection*{c}

It will take 4 seconds plus a pseudo- delay of 10 ms to fully transmit the entire message.
That is 4.01 seconds.
This is nearly one-third of the non-segmented message.
This makes sense as segmenting will allow for pipelining and more throughput.
\subsection*{d}

One reason to use message segmentation is that all the segments don't need to be sent down the same paths.
They can be spread out so that they don't cause congestion in one part of the internet.
Also, segmentation means that even if some of the packets are lost, most of the data will most likely be received by the client.

\subsection*{e}

A drawback of smessage segmentation is that it requires a lot more setup than non-segmented messages. 
Each packet must contain all the information about where it needs to go.
Also the packets will need to be reassembled, which will be much harder if the packets arrive out of order.
\end{document}